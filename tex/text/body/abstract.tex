\begin{abstract}
Gene duplications have been shown to promote evolvability in both biological and computational systems.
We use digital evolution via the Avida platform to explore \textit{why} and \textit{how} gene duplications facilitate adaptive evolution.
Are duplications valuable because they inflate the effective mutation rate and increase genetic variation?
Or is their benefit tied to the clustering of mutations, which inserts genetic material that evolution can readily co-opt to form longer, more complex genomes?
Alternatively, does the key lie in the information being duplicated—specifically, whether preserving the full structure of duplicated code is necessary, or if rearranged functional building blocks suffice?
By testing a variety of partial analogs of gene duplication within Avida, we confirm that the efficacy of gene duplications depends on the duplicated information content remaining intact.
Our experiments reveal that while simply increasing the amount of raw genetic material can facilitate the evolution of simple traits, the duplication of existing genome information uniquely potentiates the emergence of complex traits.
Tracing the lineage history of individual genome sites, we find that duplicated genome regions to be disproportionately potentiated to later code for novel complex functions, consistent with neofunctionalization theory.
Taking place \textit{in silico} within a study system that may be arbitrarily manipulated and exactly observed, reported findings help to deepen and systematize our understanding of the evolutionary consequences of gene duplications.
\end{abstract}
