\begin{abstract}
Gene duplication events are widely recognized as a key factor in enabling the evolution of organismal complexity.
Mirroring broader debate around adaptation- versus contingency-driven explanations for the evolutionary origins of biological complexity, gene duplication outcomes are typically framed in terms of neo- and sub-functionalization scenarios.
In the former, duplicated genetic material catalyzes novel functionality; in the latter, it is instead co-opted in elaborating existing functionality.
Although examples of both scenarios are widespread in natural history, practical constraints have restricted direct experimental study of the relationship between gene duplication and organismal complexity.
Using the Avida platform for digital evolution, we show that while simply expanding genome size can promote the emergence of simple adaptive traits, gene duplication uniquely facilitates the \textit{de novo} evolution of complex adaptive phenotypes.
Tracing the ancestry of individual genetic sites, we find coding sites for novel phenotypic traits to be concentrated in previously-duplicated genomic regions.
Finally, we next harness the unique capabilies of \textit{in silico} model systems to compare evolutionary outcomes under degraded variants of full slip-duplication.
This ablative analysis confirms that observed adaptive potentiation indeed arises from duplication of existing genetic information.
In contrast to purely neutral framings of biological complexity,
% whereby increases in complexity reflect an asymmetrical random walk upwards from a lower threshold,
our results support the plausibility of hypotheses proposing gene duplication events as an enabling factor promoting increases in organismal complexity via adaptive traits.
\end{abstract}
