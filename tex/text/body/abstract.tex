\begin{abstract}
Gene duplications are a well‐recognized engine of evolutionary innovation, yet questions remain around the mechanisms by which they act to enhance adaptive potential.
Here, we employ the Avida digital evolution platform to systematically dissect how different facets of gene duplication influence adaptive evolution.
By implementing a series of mutation operators that mimic various types of duplicative events --- from exact duplications that conserve both sequence content and structural organization to modifications that inject randomized genetic material --- we differentiate the benefits to adaptive evolution from increased mutational supply, clustering of mutational locality, and the propagation of existing genetic material.
Our results reveal that while simply expanding genome size can promote the emergence of simple adaptive traits, it is the preservation of existing genetic information through exact duplication that uniquely facilitates the evolution of complex phenotypes.
Tracing the ancestry of individual genetic sites, we find coding sites for novel phenotypic traits to be concentrated in previously-duplicated genomic regions, suggesting that duplications may act to potentiate discovery of complex adaptive traits.
By leveraging an \textit{in silico} study system that may be arbitrarily manipulated and exactly observed, reported findings help to deepen and systematize our understanding of the evolutionary consequences of gene duplications.
\end{abstract}
