\section{Introduction} \label{sec:introduction}

Genetic surveys and evolution experiments have made available an increasingly detailed picture of how duplicative processes shape genetic information.
In experimental yeast systems with harsh conditions, adaptations resulting from aneuploidy confer superior fitness \citep{Pavelka:2010}.
Similarly, cancer cells are commonly observed in polyaneuploid states, and the drastic genetic changes from duplications and deletions could play a large role in creating resistance mechanisms \citep{polyaneuploidCancer}.
Concrete phenotypic effects have been directly attributed to changes in copy count, such as the short leg length characteristic of dachshunds arising from an additional copy of the FGF4 gene \citep{dachshundGeneCopyNum}.

Duplications in genetic material range from repeats of gene fragments to whole genome duplication.
By providing new genetic material, these events are thought to introduce potential for further evolutionary modification.
%All of these processes result in %share the common feature of
%an increase in genome size, but more importantly, genetic material that is likely more structured and meaningful than random insertions. %, especially when the duplicated sequence already encoded for one or more traits. %with encoded information.
As such, duplication is understood to provide an essential foundation for genetic variation \citep{Zhang:2003fw,Crow:2006role,Magadum:2013wu}.
Indeed, gene duplication has been shown to promote adaptive evolution in both biological and simulation models \citep{Hu:2010ea}.
% AML (2025): I shortened the above definition of evolvability for succintness (folks should make sure they still agree with it). Even better, if we could streamline the sentence to not have a definition interrupt the flow.
% MAM Done; I've been scraping out most discussion of evolvability in favor of just discussion "promotion of adaptive evolution"

A striking example of gene duplication leading to new adaptations
comes from the Long-Term Evolution Experiment in \textit{Escherichia coli}, where a duplication in one population broadened expression of key metabolic machinery to previously inhibitive conditions \citep{blount_genomic_2012}.
As a result, the population tapped into a previously inaccessible carbon resource, resulting in a 7-fold increase in population size.
In addition to such anecdotal case studies, large-scale genomic studies have discovered cases where a strikingly high fraction of the genes in an organism show evidence of having arisen from gene duplications \citep{teichmann_structural_1998,Teichmann:2004cz}.
Comparative studies have associated duplication events early in natural history with increases in genetic robustness and evolutionary innovation \citep{wagner_gene_2008}.

The prominent role of gene duplications in biological evolution has inspired incorporation of analogous mechanisms in evolution-inspired optimization algorithms \citep{Ryan:1998gm,Sawai:1999genetic,Sawai:2000comparative,Schmitt:2005bc}.
Notably, in genetic programming, gene duplication and deletion operators have been shown to increase program evolvability and yield simpler evolved solutions \citep{Koza:1995fr}.
In work evolving neural controllers for robots, enabling module duplication was found to increase functional specialization in network modules \citep{Calabretta:1998vh,Calabretta:2000tl}.

Given the evidence that gene duplication can facilitate adaptive evolution in both computational and natural systems, we use a digital evolution approach to explore \textit{why}.
That is, what mechanistic aspects of gene duplications promote adaptive evolution?

One question is how fidelity of duplicated material influences subsequent evolutionary outcomes.
Exact duplications can result in functionally redundant genes that can increase the mutational robustness of a genotype \citep{Crow:2006role} or allow the organism to produce additional gene product \citep{Zhang:2003fw}.
If a highly constrained genetic sequence is duplicated, one copy can potentially mutate more freely and produce new functionality (i.e., neofunctionalization) \citep{Zhang:2003fw,Wagner:2003fk}.
Alternatively, subfunctionalization may occur where both gene copies diverge from the ancestral gene state, specializing in complementary aspects of the ancestral gene's functionality \citep{Zhang:2003fw}.

It is also possible that ``side effects'' of gene duplication may contribute to adaptive evolution, such as effects in increasing effective mutation rate, localized clustering of sequence changes, and increases in genome length.
Due to their inherent co-occurrence, these aspects of gene duplication are nearly impossible to disentangle in field or laboratory studies.

Interest in duplication of genetic material as an evolutionary catalyst dates well before modern understanding of how genetic information is stored and processed in biological organisms \citep{Metz:chromosomeDuplication1947}.
Today, digital systems provide an opportunity to model sophisticated evolutionary processes in a framework where they can be observed in complete detail.
In this work, we leveraged the Avida digital evolution platform \citep{ofria2004avida}, to systematically investigate how mechanistic aspects of gene duplication influence evolutionary outcomes.
% AML (2025): These next two sentences were redundant. Added alternative bullet points above.
% We introduce five slip mutation operators to tease apart the specific components of a gene duplication.
% By observing how each of our slip mutation operators affects the evolution of digital organisms in different contexts, we are able to isolate which aspects of gene duplications are most important for promoting evolvability.

\subsection{Major Results}

We found local slip mutations of intact regions to be the most effective configuration of gene duplication in facilitating evolution on the Logic-9 task set within the Avida platform.
In particular, we found that --- compared to control experiments with long genome sizes --- gene duplication uniquely promoted the evolution of complex adaptive traits.
We further found that the raw material created by slip duplication plays a potentiating role in the evolution of complex traits.
Specifically, we identified that slip-duplicated regions are significantly more likely to serve as coding sites for new traits when they first appear.
Consistent with expectations under neofunctionalization theory, however, we did not observe potentiation effects of slip duplication on the evolution of very simple traits that did not involve building block components -- only on those that require multiple components.

Finally, we assessed the consequences of slip duplication on genome architecture.
One possible expectation is that gene duplication might accelerate growth in genetic brittleness by providing raw material for contingent complexity to arise, as drift effects wear away the redundancies introduced by duplication.
Contrary to this expectation, we found that the rate of genome complexity growth was similar to that observed in control experiments.
However, we observed a significant increase in the accumulation rate of net active and vestigial coding material in genomes under slip duplication.
To understand this phenomenon, we tested the immediate effects of slip duplication on genome brittleness.
We found that, on average, neutral slip duplications decrease, rather than increase, the number of critical coding sites in a genome.
These results align with our observed increase in vestigial coding material in genomes.
These brittleness-reducing effects appear to be counteracted by other factors, resulting in a similar overall trajectory of genome complexity between slip duplication treatments and controls.
